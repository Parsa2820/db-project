\documentclass{article}

\usepackage{fancyhdr}
\usepackage{graphicx}

\usepackage{xepersian}
\settextfont{XB Zar}

\setlength{\parindent}{0pt}

\title{
\includegraphics[width=0.4\textwidth]{sharif.png}\\
\normalsize{دانشکده مهندسی کامپیوتر}\\
\vspace{1cm}
	
\huge{طراحی پایگاه‌داده}
\\ \vspace{.8cm}
\Large{پیشنهاد نیازمندی‌های پروژه}
}

\author{
\\
دکتر امینی
\\ \vspace{.4cm}
\\
  سارا آذرنوش       ---      98170668
\\ \vspace{0.2cm} \\
  سید ارشان دلیلی       ---      98105751
\\ \vspace{0.2cm} \\
  پارسا محمدیان       ---      98102284
\\ \vspace{.4cm}
}

\date{\today}

\begin{document}

\clearpage
\maketitle
\thispagestyle{empty}

\newpage

\clearpage
\pagestyle{fancy}
\lhead{طراحی پایگاه‌داده}

\rhead{پیشنهاد نیازمندی‌های پروژه}

\tableofcontents

\newpage

\setcounter{page}{1}

\section{موضوع پروژه}
بانک الکترونیکی

\section{نیازمندی‌های پروژه}

    \subsection{موجودیت‌های سیستم}

        \subsubsection{شخص حقیقی}
            ویژگی‌ها
            \begin{itemize}
                \item نام
                \item نام خانوادگی
                \item نام پدر
                \item کد ملی
                \item تاریخ تولد
                \item شماره تلفن همراه
                \item شماره تلفن ثابت
                \item آدرس
            \end{itemize}

        \subsubsection{کارمند}
            ویژگی‌ها
            \begin{itemize}
                \item نام
                \item نام خانوادگی
                \item نام پدر
                \item کد ملی
                \item تاریخ تولد
                \item شماره تلفن همراه
                \item شماره تلفن ثابت
                \item آدرس
                \item سمت (مدیر، کارشناس، پشتیبانی، و...)
                \item حقوق
                \item ساعت کاری
                \item تاریخ شروع به کار
            \end{itemize}

        \subsubsection{حساب کاربری}
            ویژگی‌ها
            \begin{itemize}
                \item نام کاربری
                \item رمز عبور
                \item ایمیل (فراموشی رمز عبور و اعلانات)
                \item شماره تلفن همراه (پیامک اعلانات و رمز عبور یک‌بار مصرف)
            \end{itemize}

        \subsubsection{حساب بانکی}
            ویژگی‌ها
            \begin{itemize}
                \item شماره
                \item شبا
                \item موجودی
                \item موجودی قابل برداشت
                \item سابقه حساب
                \item تاریخ افتتاح
                \item صاحب حساب
                \item حذف شده/فعال
                \item مبلغ تراکنش‌های امروز
            \end{itemize}
            انواع
            \begin{itemize}
                \item قرض‌الحسنه پس‌انداز
                \item سپرده
            \end{itemize}

        \subsubsection{کارت}
            ویژگی‌ها
            \begin{itemize}
                \item شماره کارت
                \item تاریخ انقضا
                \item رمز اول
                \item رمز دوم
                \item \lr{CVV1}
                \item \lr{CVV2}
                \item فعال/غیرفعال
            \end{itemize}

        \subsubsection{تراکنش}
            ویژگی‌ها
            \begin{itemize}
             \item شماره پیگیری
                \item تاریخ
                \item مبلغ
                \item مبدا
                \item مقصد
            \end{itemize}
            انواع
            \begin{itemize}
                \item واریز
                \item برداشت
                \item انتقال وجه
                \item پرداخت قبض (شامل شارژ و \dots)
                \item خرید
            \end{itemize}
            
        \subsubsection{درخواست کاربر}
            ویژگی‌ها
            \begin{itemize}
             \item شماره پیگیری
                \item تاریخ
                \item کاربر ایجادکننده
                \item توضیحات
                \item وضعیت (در جریان، پذیرفته شده، رد شده)
                \item جواب پشتیبانی
                \item کارمند پشتیبان
            \end{itemize}


    \subsection{نیازهای کاربران از سیستم}
        \begin{itemize}
            \item اشخاص حقوقی باید بتوانند حساب کاربری و با استفاده از آن حساب بانکی موردنظر را بسازند.
            \item مشتری باید بتواند انواع تراکنش‌های مجاز را انجام بدهد.
            \item مشتری باید بتواند درخواست استفاده از خدمات مانند صدور کارت را بدهد و پس از بررسی در صورت تایید صلاحیت، خدمات اعطا شود.
            \item مشتری باید بتواند سابقه تراکنش‌ها و اطلاعات حساب‌ خود را ببیند.
            \item مشتری باید بتواند رمز عبور حساب خود و رمزهای اول و دوم کارت خود را تعویض کند و در صورت فراموشی رمز عبور حساب کاربری، آن را بازیابی کند.
            \item مشتری باید بتواند حساب‌ خود را ببندد.
            \item مشتری باید بتواند کارت‌ خود را موقتاً یا دائمی غیر فعال کند.
            \item مشتری باید بتواند به‌طور مستمر با پشتیبانی در تماس باشد.
            \item کارمند باید بتواند اطلاعات مورد نیاز مالی مشتری را ببیند. همچنین کارمند باید بتواند 
            داده را با ترتیب‌های مختلف و جستجوهای مختلف (بازه‌ای، دقیق، ...) بازیابی کند.
            \item کارمند باید به درخواست خدمات مشتری رسیدگی کند.
            \item سود حساب‌های بانکی سپرده که مشمول سود هستند، به صورت ماهانه محاسبه و پرداخت شوند.
            \item مدیریت باید بتواند از تمام اطلاعات گزارش بگیرد و بر هر اساسی مرتب سازی کند. همچنین مدیر باید 
            به صورت‌های مختلف (مانند کارمند) جستجو انجام دهد.
            \item محدودیت‌هایی مانند سقف مجاز تراکنش، کف تراکنش مجاز، کف موجودی و ... باید لحاظ شوند. 
        \end{itemize}

\textbf{توضیح در مورد پیشنهادهای نیازمندی‌ها:}
طبق مکاتبات انجام شده با دستیار آموزشی محترم، بلوبانک یک نمونه واقعی از بانک الکترونیکی 
محسوب می‌شود. به همین دلیل طبق پیشنهاد دستیار آموزشی مبنی بر نزدیک ساختن پروژه به نمونه دنیای واقعی،
نیازمندی‌ها مطابق بلوبانک نوشته شده است. 
    
\section{فاز ۱}
فاز ۱


\section{فاز ۲}
فاز ۲

\section{فاز ۳}
	\subsection{دامنه‌های صفات، محدودیت‌های جامعیتی، رهانا و نرمال‌}
ابتدا



\end{document}