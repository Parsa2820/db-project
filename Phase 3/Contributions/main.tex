\documentclass{article}

\usepackage{fancyhdr}
\usepackage{graphicx}

\usepackage{xepersian}
\settextfont{XB Zar}

\setlength{\parindent}{0pt}

\title{
\includegraphics[width=0.4\textwidth]{sharif.png}\\
\normalsize{دانشکده مهندسی کامپیوتر}\\
\vspace{1cm}
	
\huge{طراحی پایگاه‌داده}
\\ \vspace{.8cm}
\Large{داک مشارکت پروژه}
}

\author{
\\
دکتر امینی
\\ \vspace{.4cm}
\\
  سارا آذرنوش       ---      98170668
\\ \vspace{0.2cm} \\
  سید ارشان دلیلی       ---      98105751
\\ \vspace{0.2cm} \\
  پارسا محمدیان       ---      98102284
\\ \vspace{.4cm}
}

\date{\today}

\begin{document}

\clearpage
\maketitle
\thispagestyle{empty}

\newpage

\clearpage
\pagestyle{fancy}
\lhead{طراحی پایگاه‌داده}

\rhead{مشارکت اعضا گروه}

\tableofcontents

\newpage

\setcounter{page}{1}

\section{موضوع پروژه}
بانک الکترونیکی

\section{بخش نیازمندی‌های پروژه}
در بخش نیازمندی‌های پروژه، هر سه نفر از اعضای گروه مشارکت یکسانی در بررسی موجودیت‌ها و هم‌چنین نیازمندی‌های احتمالی سیستم مورد نیاز برای پیاده‌سازی یک بانک الکترونیکی (مانند بلوبانک) داشتند.

\section{فاز ۱}
در فاز ۱ پروژه، هر سه نفر از اعضای پروژه، نقش یکسانی در کشیدن نمودار EER پروژه داشتند. این نمودار بر بستر draw.io و با مشارکت هر سه عضو گروه تهیه شد. هم‌چنین اصلاحاتی در نیازمندی‌های پروژه اعمال شد.

\section{فاز ۲}
در فاز ۲ پروژه، هر سه نفر در بخش‌های مختلف فاز مشارکت داشتند. آقای محمدیان و خانم آذرنوش بیش‌تر بر روی تهیه جدول‌ها و آقای دلیلی بیش‌تر بر روی تهیه داکیومنتیشن برای ابزار مورد استفاده (PostgreSQL) مشغول بودند. (البته تمام اعضا بر روی هر دو بخش کار می‌کردند.)

\section{فاز ۳}

در فاز ۳ پروژه، تمام اعضا در بخش‌های مختلف مانند به دست آوردن دامنه صفات جدول‌ها، محدودیت‌های جامعیتی، تریگرها و نرمال‌سازی وقت گذاشتند و با هم‌فکری هم موارد خواسته شده را به دست آوردند. سپس با همکاری یک‌دیگر جدول‌ها و کوئری‌های مورد نیاز SQL را نوشتند و بر بستر django، یک سرویس با امکان دریافت مطلوبات کاربر و نشان دادن خروجی را پیاده‌سازی کردند. لازم به ذکر است تمام اعضای گروه تقریباً در انجام تمام کارها مشارکت داشتند.


\end{document}